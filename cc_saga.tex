% Options for packages loaded elsewhere
\PassOptionsToPackage{unicode}{hyperref}
\PassOptionsToPackage{hyphens}{url}
%
\documentclass[
]{article}
\usepackage{lmodern}
\usepackage{amssymb,amsmath}
\usepackage{ifxetex,ifluatex}
\ifnum 0\ifxetex 1\fi\ifluatex 1\fi=0 % if pdftex
  \usepackage[T1]{fontenc}
  \usepackage[utf8]{inputenc}
  \usepackage{textcomp} % provide euro and other symbols
\else % if luatex or xetex
  \usepackage{unicode-math}
  \defaultfontfeatures{Scale=MatchLowercase}
  \defaultfontfeatures[\rmfamily]{Ligatures=TeX,Scale=1}
\fi
% Use upquote if available, for straight quotes in verbatim environments
\IfFileExists{upquote.sty}{\usepackage{upquote}}{}
\IfFileExists{microtype.sty}{% use microtype if available
  \usepackage[]{microtype}
  \UseMicrotypeSet[protrusion]{basicmath} % disable protrusion for tt fonts
}{}
\makeatletter
\@ifundefined{KOMAClassName}{% if non-KOMA class
  \IfFileExists{parskip.sty}{%
    \usepackage{parskip}
  }{% else
    \setlength{\parindent}{0pt}
    \setlength{\parskip}{6pt plus 2pt minus 1pt}}
}{% if KOMA class
  \KOMAoptions{parskip=half}}
\makeatother
\usepackage{xcolor}
\IfFileExists{xurl.sty}{\usepackage{xurl}}{} % add URL line breaks if available
\IfFileExists{bookmark.sty}{\usepackage{bookmark}}{\usepackage{hyperref}}
\hypersetup{
  pdftitle={R Notebook},
  hidelinks,
  pdfcreator={LaTeX via pandoc}}
\urlstyle{same} % disable monospaced font for URLs
\usepackage[margin=1in]{geometry}
\usepackage{color}
\usepackage{fancyvrb}
\newcommand{\VerbBar}{|}
\newcommand{\VERB}{\Verb[commandchars=\\\{\}]}
\DefineVerbatimEnvironment{Highlighting}{Verbatim}{commandchars=\\\{\}}
% Add ',fontsize=\small' for more characters per line
\usepackage{framed}
\definecolor{shadecolor}{RGB}{248,248,248}
\newenvironment{Shaded}{\begin{snugshade}}{\end{snugshade}}
\newcommand{\AlertTok}[1]{\textcolor[rgb]{0.94,0.16,0.16}{#1}}
\newcommand{\AnnotationTok}[1]{\textcolor[rgb]{0.56,0.35,0.01}{\textbf{\textit{#1}}}}
\newcommand{\AttributeTok}[1]{\textcolor[rgb]{0.77,0.63,0.00}{#1}}
\newcommand{\BaseNTok}[1]{\textcolor[rgb]{0.00,0.00,0.81}{#1}}
\newcommand{\BuiltInTok}[1]{#1}
\newcommand{\CharTok}[1]{\textcolor[rgb]{0.31,0.60,0.02}{#1}}
\newcommand{\CommentTok}[1]{\textcolor[rgb]{0.56,0.35,0.01}{\textit{#1}}}
\newcommand{\CommentVarTok}[1]{\textcolor[rgb]{0.56,0.35,0.01}{\textbf{\textit{#1}}}}
\newcommand{\ConstantTok}[1]{\textcolor[rgb]{0.00,0.00,0.00}{#1}}
\newcommand{\ControlFlowTok}[1]{\textcolor[rgb]{0.13,0.29,0.53}{\textbf{#1}}}
\newcommand{\DataTypeTok}[1]{\textcolor[rgb]{0.13,0.29,0.53}{#1}}
\newcommand{\DecValTok}[1]{\textcolor[rgb]{0.00,0.00,0.81}{#1}}
\newcommand{\DocumentationTok}[1]{\textcolor[rgb]{0.56,0.35,0.01}{\textbf{\textit{#1}}}}
\newcommand{\ErrorTok}[1]{\textcolor[rgb]{0.64,0.00,0.00}{\textbf{#1}}}
\newcommand{\ExtensionTok}[1]{#1}
\newcommand{\FloatTok}[1]{\textcolor[rgb]{0.00,0.00,0.81}{#1}}
\newcommand{\FunctionTok}[1]{\textcolor[rgb]{0.00,0.00,0.00}{#1}}
\newcommand{\ImportTok}[1]{#1}
\newcommand{\InformationTok}[1]{\textcolor[rgb]{0.56,0.35,0.01}{\textbf{\textit{#1}}}}
\newcommand{\KeywordTok}[1]{\textcolor[rgb]{0.13,0.29,0.53}{\textbf{#1}}}
\newcommand{\NormalTok}[1]{#1}
\newcommand{\OperatorTok}[1]{\textcolor[rgb]{0.81,0.36,0.00}{\textbf{#1}}}
\newcommand{\OtherTok}[1]{\textcolor[rgb]{0.56,0.35,0.01}{#1}}
\newcommand{\PreprocessorTok}[1]{\textcolor[rgb]{0.56,0.35,0.01}{\textit{#1}}}
\newcommand{\RegionMarkerTok}[1]{#1}
\newcommand{\SpecialCharTok}[1]{\textcolor[rgb]{0.00,0.00,0.00}{#1}}
\newcommand{\SpecialStringTok}[1]{\textcolor[rgb]{0.31,0.60,0.02}{#1}}
\newcommand{\StringTok}[1]{\textcolor[rgb]{0.31,0.60,0.02}{#1}}
\newcommand{\VariableTok}[1]{\textcolor[rgb]{0.00,0.00,0.00}{#1}}
\newcommand{\VerbatimStringTok}[1]{\textcolor[rgb]{0.31,0.60,0.02}{#1}}
\newcommand{\WarningTok}[1]{\textcolor[rgb]{0.56,0.35,0.01}{\textbf{\textit{#1}}}}
\usepackage{graphicx,grffile}
\makeatletter
\def\maxwidth{\ifdim\Gin@nat@width>\linewidth\linewidth\else\Gin@nat@width\fi}
\def\maxheight{\ifdim\Gin@nat@height>\textheight\textheight\else\Gin@nat@height\fi}
\makeatother
% Scale images if necessary, so that they will not overflow the page
% margins by default, and it is still possible to overwrite the defaults
% using explicit options in \includegraphics[width, height, ...]{}
\setkeys{Gin}{width=\maxwidth,height=\maxheight,keepaspectratio}
% Set default figure placement to htbp
\makeatletter
\def\fps@figure{htbp}
\makeatother
\setlength{\emergencystretch}{3em} % prevent overfull lines
\providecommand{\tightlist}{%
  \setlength{\itemsep}{0pt}\setlength{\parskip}{0pt}}
\setcounter{secnumdepth}{-\maxdimen} % remove section numbering

\title{R Notebook}
\author{}
\date{\vspace{-2.5em}}

\begin{document}
\maketitle

\begin{Shaded}
\begin{Highlighting}[]
\KeywordTok{library}\NormalTok{(tidyverse)}
\end{Highlighting}
\end{Shaded}

\begin{verbatim}
## -- Attaching packages -------------------------------------- tidyverse 1.3.0 --
\end{verbatim}

\begin{verbatim}
## v ggplot2 3.3.0     v purrr   0.3.4
## v tibble  3.0.1     v dplyr   0.8.5
## v tidyr   1.0.3     v stringr 1.4.0
## v readr   1.3.1     v forcats 0.5.0
\end{verbatim}

\begin{verbatim}
## -- Conflicts ----------------------------------------- tidyverse_conflicts() --
## x dplyr::filter() masks stats::filter()
## x dplyr::lag()    masks stats::lag()
\end{verbatim}

\begin{Shaded}
\begin{Highlighting}[]
\KeywordTok{library}\NormalTok{(dplyr)}
\KeywordTok{library}\NormalTok{(ggplot2)}
\CommentTok{# This sets the size of plots to a good default.}
\KeywordTok{options}\NormalTok{(}\DataTypeTok{repr.plot.width =} \DecValTok{5}\NormalTok{, }\DataTypeTok{repr.plot.height =} \DecValTok{4}\NormalTok{)}

\KeywordTok{getwd}\NormalTok{()}
\end{Highlighting}
\end{Shaded}

\begin{verbatim}
## [1] "C:/Users/dolug/Documents/R/Candy Crush Difficulty"
\end{verbatim}

\begin{Shaded}
\begin{Highlighting}[]
\NormalTok{candy_crush <-}\StringTok{ }\KeywordTok{read.csv}\NormalTok{(}\StringTok{"candy_crush.csv"}\NormalTok{)}
\KeywordTok{head}\NormalTok{(candy_crush)}
\end{Highlighting}
\end{Shaded}

\begin{verbatim}
##                          player_id         dt level num_attempts num_success
## 1 6dd5af4c7228fa353d505767143f5815 2014-01-04     4            3           1
## 2 c7ec97c39349ab7e4d39b4f74062ec13 2014-01-01     8            4           1
## 3 c7ec97c39349ab7e4d39b4f74062ec13 2014-01-05    12            6           0
## 4 a32c5e9700ed356dc8dd5bb3230c5227 2014-01-03    11            1           1
## 5 a32c5e9700ed356dc8dd5bb3230c5227 2014-01-07    15            6           0
## 6 b94d403ac4edf639442f93eeffdc7d92 2014-01-01     8            8           1
\end{verbatim}

\begin{Shaded}
\begin{Highlighting}[]
\CommentTok{#the amount of unique ids}
\KeywordTok{print}\NormalTok{(}\StringTok{"Number of players:"}\NormalTok{)}
\end{Highlighting}
\end{Shaded}

\begin{verbatim}
## [1] "Number of players:"
\end{verbatim}

\begin{Shaded}
\begin{Highlighting}[]
\KeywordTok{length}\NormalTok{(}\KeywordTok{unique}\NormalTok{(candy_crush}\OperatorTok{$}\NormalTok{player_id))}
\end{Highlighting}
\end{Shaded}

\begin{verbatim}
## [1] 6814
\end{verbatim}

\begin{Shaded}
\begin{Highlighting}[]
\CommentTok{#the time frame we will be evaluating}
\KeywordTok{print}\NormalTok{(}\StringTok{"Period for which we have data:"}\NormalTok{)}
\end{Highlighting}
\end{Shaded}

\begin{verbatim}
## [1] "Period for which we have data:"
\end{verbatim}

\begin{Shaded}
\begin{Highlighting}[]
\KeywordTok{range}\NormalTok{(candy_crush}\OperatorTok{$}\NormalTok{dt)}
\end{Highlighting}
\end{Shaded}

\begin{verbatim}
## [1] "2014-01-01" "2014-01-07"
\end{verbatim}

\begin{Shaded}
\begin{Highlighting}[]
\CommentTok{#calculating level difficulty}
\NormalTok{difficulty <-}\StringTok{ }\NormalTok{candy_crush }\OperatorTok\StringTok{ }\KeywordTok{group_by}\NormalTok{(level) }\OperatorTok\StringTok{ }\KeywordTok{summarise}\NormalTok{(}
  \DataTypeTok{attempts =} \KeywordTok{sum}\NormalTok{(num_attempts), }\DataTypeTok{success =} \KeywordTok{sum}\NormalTok{(num_success)) }\OperatorTok\StringTok{ }
\StringTok{  }\KeywordTok{mutate}\NormalTok{(}\DataTypeTok{p_success =}\NormalTok{ success}\OperatorTok{/}\NormalTok{attempts)}

\KeywordTok{print}\NormalTok{(difficulty)}
\end{Highlighting}
\end{Shaded}

\begin{verbatim}
## # A tibble: 15 x 4
##    level attempts success p_success
##    <int>    <int>   <int>     <dbl>
##  1     1     1322     818    0.619 
##  2     2     1285     666    0.518 
##  3     3     1546     662    0.428 
##  4     4     1893     705    0.372 
##  5     5     6937     634    0.0914
##  6     6     1591     668    0.420 
##  7     7     4526     614    0.136 
##  8     8    15816     641    0.0405
##  9     9     8241     670    0.0813
## 10    10     3282     617    0.188 
## 11    11     5575     603    0.108 
## 12    12     6868     659    0.0960
## 13    13     1327     686    0.517 
## 14    14     2772     777    0.280 
## 15    15    30374    1157    0.0381
\end{verbatim}

\begin{Shaded}
\begin{Highlighting}[]
\CommentTok{#plotting the level difficulty profile}
\KeywordTok{ggplot}\NormalTok{(difficulty,}\KeywordTok{aes}\NormalTok{(level,p_success)) }\OperatorTok{+}
\StringTok{  }\KeywordTok{geom_line}\NormalTok{() }\OperatorTok{+}\StringTok{ }\KeywordTok{scale_x_continuous}\NormalTok{(}\DataTypeTok{breaks =} \DecValTok{1}\OperatorTok{:}\DecValTok{15}\NormalTok{) }\OperatorTok{+}
\StringTok{  }\KeywordTok{scale_y_continuous}\NormalTok{(}\DataTypeTok{label =}\NormalTok{ scales}\OperatorTok{::}\NormalTok{percent) }\OperatorTok{+}
\StringTok{  }\KeywordTok{ylab}\NormalTok{(}\StringTok{"Probability of beating Level"}\NormalTok{) }\OperatorTok{+}
\StringTok{  }\KeywordTok{ggtitle}\NormalTok{(}\StringTok{"Level Difficulty"}\NormalTok{) }\OperatorTok{+}
\StringTok{  }\KeywordTok{theme}\NormalTok{(}\DataTypeTok{plot.title =} \KeywordTok{element_text}\NormalTok{(}\DataTypeTok{hjust =} \FloatTok{0.5}\NormalTok{))}
\end{Highlighting}
\end{Shaded}

\includegraphics{cc_saga_files/figure-latex/unnamed-chunk-1-1.pdf}

\begin{Shaded}
\begin{Highlighting}[]
\CommentTok{#to identity "hard" levels by defining a threshold}
\CommentTok{#that levels with p_success < 10% are considered hard.}
\KeywordTok{ggplot}\NormalTok{(difficulty,}\KeywordTok{aes}\NormalTok{(level,p_success)) }\OperatorTok{+}
\StringTok{  }\KeywordTok{geom_line}\NormalTok{() }\OperatorTok{+}\StringTok{ }\KeywordTok{scale_x_continuous}\NormalTok{(}\DataTypeTok{breaks =} \DecValTok{1}\OperatorTok{:}\DecValTok{15}\NormalTok{) }\OperatorTok{+}
\StringTok{  }\KeywordTok{scale_y_continuous}\NormalTok{(}\DataTypeTok{label =}\NormalTok{ scales}\OperatorTok{::}\NormalTok{percent) }\OperatorTok{+}
\StringTok{  }\KeywordTok{ylab}\NormalTok{(}\StringTok{"Probability of beating Level"}\NormalTok{) }\OperatorTok{+}
\StringTok{  }\KeywordTok{ggtitle}\NormalTok{(}\StringTok{"Level Difficulty"}\NormalTok{) }\OperatorTok{+}
\StringTok{  }\KeywordTok{theme}\NormalTok{(}\DataTypeTok{plot.title =} \KeywordTok{element_text}\NormalTok{(}\DataTypeTok{hjust =} \FloatTok{0.5}\NormalTok{)) }\OperatorTok{+}
\StringTok{  }\KeywordTok{geom_point}\NormalTok{() }\OperatorTok{+}
\StringTok{  }\KeywordTok{geom_hline}\NormalTok{(}\DataTypeTok{yintercept=}\FloatTok{0.1}\NormalTok{, }\DataTypeTok{linetype=}\StringTok{'dashed'}\NormalTok{, }\DataTypeTok{color=}\StringTok{'red'}\NormalTok{)}
\end{Highlighting}
\end{Shaded}

\includegraphics{cc_saga_files/figure-latex/unnamed-chunk-1-2.pdf}

\begin{Shaded}
\begin{Highlighting}[]
\CommentTok{#computing standard error of p_success of each level}
\NormalTok{error_fn <-}\StringTok{ }\NormalTok{difficulty }\OperatorTok\StringTok{ }\KeywordTok{mutate}\NormalTok{(}\DataTypeTok{error =} \KeywordTok{sqrt}\NormalTok{(p_success }\OperatorTok{*}\StringTok{ }\NormalTok{(}\DecValTok{1} \OperatorTok{-}\StringTok{ }\NormalTok{p_success)}\OperatorTok{/}\NormalTok{attempts))}
\NormalTok{(error_fn)}
\end{Highlighting}
\end{Shaded}

\begin{verbatim}
## # A tibble: 15 x 5
##    level attempts success p_success   error
##    <int>    <int>   <int>     <dbl>   <dbl>
##  1     1     1322     818    0.619  0.0134 
##  2     2     1285     666    0.518  0.0139 
##  3     3     1546     662    0.428  0.0126 
##  4     4     1893     705    0.372  0.0111 
##  5     5     6937     634    0.0914 0.00346
##  6     6     1591     668    0.420  0.0124 
##  7     7     4526     614    0.136  0.00509
##  8     8    15816     641    0.0405 0.00157
##  9     9     8241     670    0.0813 0.00301
## 10    10     3282     617    0.188  0.00682
## 11    11     5575     603    0.108  0.00416
## 12    12     6868     659    0.0960 0.00355
## 13    13     1327     686    0.517  0.0137 
## 14    14     2772     777    0.280  0.00853
## 15    15    30374    1157    0.0381 0.00110
\end{verbatim}

\begin{Shaded}
\begin{Highlighting}[]
\CommentTok{#Adding std error bars}
\KeywordTok{ggplot}\NormalTok{(error_fn,}\KeywordTok{aes}\NormalTok{(level,p_success)) }\OperatorTok{+}
\StringTok{  }\KeywordTok{geom_line}\NormalTok{() }\OperatorTok{+}\StringTok{ }\KeywordTok{scale_x_continuous}\NormalTok{(}\DataTypeTok{breaks =} \DecValTok{1}\OperatorTok{:}\DecValTok{15}\NormalTok{) }\OperatorTok{+}
\StringTok{  }\KeywordTok{scale_y_continuous}\NormalTok{(}\DataTypeTok{labels =}\NormalTok{ scales}\OperatorTok{::}\NormalTok{percent) }\OperatorTok{+}
\StringTok{  }\KeywordTok{ylab}\NormalTok{(}\StringTok{"Probability of beating Level"}\NormalTok{) }\OperatorTok{+}
\StringTok{  }\KeywordTok{ggtitle}\NormalTok{(}\StringTok{"Level Difficulty"}\NormalTok{) }\OperatorTok{+}
\StringTok{  }\KeywordTok{theme}\NormalTok{(}\DataTypeTok{plot.title =} \KeywordTok{element_text}\NormalTok{(}\DataTypeTok{hjust =} \FloatTok{0.5}\NormalTok{)) }\OperatorTok{+}
\StringTok{  }\KeywordTok{geom_point}\NormalTok{() }\OperatorTok{+}
\StringTok{  }\KeywordTok{geom_hline}\NormalTok{(}\DataTypeTok{yintercept=}\FloatTok{0.1}\NormalTok{, }\DataTypeTok{linetype=}\StringTok{'dashed'}\NormalTok{) }\OperatorTok{+}
\StringTok{  }\KeywordTok{geom_errorbar}\NormalTok{(}\KeywordTok{aes}\NormalTok{(}\DataTypeTok{ymin=}\NormalTok{ p_success }\OperatorTok{-}\StringTok{ }\NormalTok{error, }\DataTypeTok{ymax =}\NormalTok{ p_success }\OperatorTok{+}\StringTok{ }\NormalTok{error), }
                \DataTypeTok{color=}\StringTok{'red'}\NormalTok{)}
\end{Highlighting}
\end{Shaded}

\includegraphics{cc_saga_files/figure-latex/unnamed-chunk-1-3.pdf}

\begin{Shaded}
\begin{Highlighting}[]
\CommentTok{# The probability of completing the episode without losing a single time}
\NormalTok{p <-}\StringTok{ }\KeywordTok{prod}\NormalTok{(difficulty}\OperatorTok{$}\NormalTok{p_success)}

\CommentTok{# Printing it out}
\NormalTok{p}
\end{Highlighting}
\end{Shaded}

\begin{verbatim}
## [1] 9.447141e-12
\end{verbatim}

\end{document}
